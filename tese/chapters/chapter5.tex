\chapter{Conclusion}%
\label{chapter:conclusion}

\begin{introduction}
The last chapter presents the conclusions of this dissertation.
\end{introduction} 


This dissertation focuses on the planning of the development of a vehicle maintenance web application for LightMobie's dealerships. 
The research on this topic reveals the importance of assuring quality of service in this market, presenting a paper to corroborate this statement.
The paper describes a study made in South Africa using the model SERVQUAL to measure the quality of service in the vehicle maintenance service.
This section also presents a web application using Laravel to increase the performance of the work at a dealership in Libya. 
The results were positive, however, the dealer's workers recommended some improvements to the system.

The development of the web application was based on the workflow of the research paper with an emphasis on quality service evaluation.
The application is also planned to integrate a recommendation from the \citet{MAS_MOTORS}, namely the introduction of an SMS service reminder. 

For the work plan, the use cases were divided into 6 sprints. 
These sprints are two weeks long and have a week interval in the middle and at the end of the development, in case of delays in a few tasks.
After the development of the application, a period is reserved for user testing and adjustments, giving the remaining time to write the dissertation's results.


\section{Future work}

- internal messaging system to comunicate between the rececionist and the mechanics
- adicionar category parts to the administrator
- allow mechanic to do tasks that aren't assigned to him
- The form of assigned task and shedule a maintenance add the date to the right (?), since it's in the left and hidden.... may suggest difficulçty to the user
- remove passsword and allow it to do like the client

