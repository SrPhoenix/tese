\chapter{Methodology}%
\label{chapter:methodology}

\begin{introduction}
This chapter presents the non-functional requirements and the use cases of the applications. 
\end{introduction} 




\section{Applications Requirements} 


To identify the requirements of the application, I used the FURPS model. 
The non-functional requirements are listed below:

\begin{itemize}
  \item Scalability – The system should be able to support users from Europe.
  \item Reliability – The system should have a high availability of 99\% and the system also should not take longer than 4 hours to recover from a failure.
  \item Performance – The system should not take longer than 2 seconds to respond.
  \item Usability – The train of a new user should not take longer than 8 hours.
  \item Supportability – The system should be supported on the web browsers Chrome, Firefox, Microsoft Edge, and Safari.
\end{itemize}

As mentioned in Chapter II, the dealership application is structured into four key user roles: receptionist, mechanic, warehouse operator, and workshop manager.
There is a also a client aplication and an administrator.
The fuctional requirements are written below:
To manage the system it is needed an administrator to accomplish the following tasks.
- Create, change, view and remove tasks types
- Create, change, view and remove dealerships
  - Add and remove vehicle types that a dealership can operate
- Create, change, view and remove vehicle parts
- Create and view dealership employee

To interact with the client it is need a rececionist. The requirements are:
- can change betwen english and portuguese
- see number of working hours in a day 
- see the working hours of a user in a day 
- Schedule a vehicle reception date with the following information
  - vehicle registration number
  - client email
  - owner name (if applicable)
  - vehicle reception date
- create a maintenance with an expected budger, the tasks that should be performed and am expected conclusion date
- see active maintenances with the following information:
  - the name of the client
  - with the vehicle registration number
  - The entity of the vehicle if applicable
  - the creation date of the maintenance
  - evaluation date
  - expected conclusion date
  - expected budget
  - Number of hours of work expected
  - the tasks planed to be perfomerd
- The rececionist should receive a notification if there is a change in the expected budget or the expected conclusion date of the maintenance
- the rececionist should be able to confirm or reject changes of the maintenance
- the rececionist should be able to cancel the maintenance
- conclude a maintenance
- Notify when a maintenance has all planed tasks concluded 

the mechanic should be able to:
- can change betwen english and portuguese
- see the tasks needed for todays date
- pause a task
- continue a task
- see the following information in a maintenance task 
  - the vehicle registration number
  - type of the task
  - vehicle parts
  - description of each step to conclude a task
  - part needed to conclude the task
  - tarefas pretendidas pelo cliente
- finalize a task
- write a comment on a task before complete it 
- see the following information in an evaluation task
  - the tasks needed to do the maintenance
  - client comment
- if the dealership does not do evaluations, when finish an evaluation task, finish the maintenance with the tasks selected

the warehouse manager should do:
- can change betwen english and portuguese
- see the Inventory of the dealership with the following information for each part:
  - part name
  - part code
  - quantity available
  - location code in the warehouse
  - description
  - part category
  - price per unit
  - quantity per group
  - minimum value to generate alarm
  - maximum value
  - quantity to generate an automatic purchase requested
  - quantity in the automatic purchase request
- it should be able to edit the information of a part type, like:
  - location code in the warehouse
  - minimum value to generate alarm
  - maximum value
  - quantity to generate an automatic purchase requested
  - quantity in the automatic purchase request
- it should be able to see the movements of the quantity of each part in time 
- it can see the suppliers with the info:
  - name
  - phonenumber
  - email
  - address
  - list of parts in the contract and the start date and end date of each of them
- create a purchase with the following information:
  - motive
  - parts
  - quantity for each part
- it can see the details of all purchase like
 - the state
 - arrival date
 - total price
 - motive
 - creation date
 - parts of the purchase with:
  - the name of the part
  - quantity in stock
  - price 
  - tasks associated with this purchase
- it can register a purchase by selecting a expected arrival date in a assigned purchase
- it can register a delay on a waiting delivery purchase by selecting a new expected arrival date
- it can finalize a waiting delivery purchase by registering the parts received

The workshop manager can do:
- can change betwen english and portuguese
- see number of working hours in a day 
- see the working hours of a user in a day 
- see the tasks that doesn't have a mechanic assigned
- assign a task to a mechanic
- see active maintenances with the following information:
  - the name of the client
  - with the vehicle registration number
  - The entity of the vehicle if applicable
  - the creation date of the maintenance
  - evaluation date
  - expected conclusion date
  - expected budget
  - Number of hours of work expected
  - the tasks planed to be perfomerd
  - the tasks done
- it can add a task to an active maintenance
- it can see completed tasks and:
  - filter by:
    - client
    - vehicle
    - date
  - see the following information:
    - the name of the client
    - with the vehicle registration number
    - The entity of the vehicle if applicable
    - the creation date of the maintenance
    - evaluation date
    - expected conclusion date
    - expected budget
    - Number of hours of work expected
    - the tasks done
    - price
  - it can export a pdf with the same information
  - see total price received from mainteance per month
  - see total hours worked per month
- assign an purchase to a operator
- see the purchases requests with the information:
  - price
  - motive
  - creation date
  - parts of the vehicle with:
    - quantity of the part
    - quantity in stock
    - price
    - maitenance tasks associated with the purchase
- reject or authorize a purchase request
- can create a supplier with:
  - name
  - phonenumber
  - email
  - address
  - list of parts in the contract with the start date, end date and price of the part for each of them
- it can see the suppliers with the info:
  - name
  - phonenumber
  - email
  - address
  - list of parts in the contract with the start date, end date and price of the part for each of them
- it can see partnerships with entities
- it can accept or reject partnerships with entities
- see a list of employees with the info:
  - email
  - name
  - dob
  - phonenumber
  - sex
  - role
- it can create a employee with :
  - email
  - name
  - dob
  - phonenumber
  - sex
  - role
  - password













The system also has requirements.
Do the repair report... (? os outros é que podem fazer download do report?)
Tasks have a sequence to be done
\item Use Case 2.5 – Making the repair report
\begin{itemize}
  \item Scenario – After vehicle maintenance.
  \item Objective – Conclude the maintenance of a vehicle.
  \item System – The Mechanic enters into the system all operations and tests carried out on the system as well as their results.
\end{itemize}
purchase has multiple parts


\section{Applications use Cases} 
I developed a set of use cases for each user.

The receptionist will be responsible for interacting with the client, this includes the vehicle check-in and check-out and user communication. 
The use cases are:

\begin{itemize}
    \item Use Case 1.1 – Maintenance Schedule
    \begin{itemize}
      \item Scenario – The client arrives at the dealership with a vehicle to be repaired.
      \item Objective – Create a new maintenance request in the system.
      \item System – The receptionist fills a form with the vehicle registration, the evaluation date, the entity associated, the client email, client notes and tasks requested by the user
    \end{itemize}
        \item Use Case 1.2 – Define maintenance details
    \begin{itemize}
      \item Scenario – The vehicle evaluation has terminated
      \item Objective – Define the budget, a conclusion date and the tasks to be performed.
      \item System – The receptionist receives a notification that the vehicle evaluation is complete and, together with the customer, determines which tasks have to be performed. The system calculates the price, and the receptionist sets a completion date.
    \end{itemize}
    \item Use Case 1.3 – Collect information about a maintenance request
    \begin{itemize}
      \item Scenario – A Client calls the dealership to ask about the maintenance of his vehicle.
      \item Objective – Visualize the maintenance information.
      \item System –  The Receptionist searches a list of maintenance requests by vehicle or customer and he can visualize the details of the selected maintenance.
    \end{itemize}
    \item Use Case 1.4 – Accept maintenance changes
    \begin{itemize}
      \item Scenario – A problem in a vehicle maintenance has occurred and the initial agreement with the client was broken, but the client accepts the changes.
      \item Objective – Accept the changes of the maintenance.
      \item System – The Receptionist goes to the maintenance details, to the section of the maintenance changes and accept the changes for that maintenance.
    \end{itemize}
    \item Use Case 1.4 – Refuse maintenance changes
    \begin{itemize}
      \item Scenario – A problem in a vehicle maintenance has occurred and the initial agreement with the client was broken and the client refuses the changes.
      \item Objective – Refuse the changes of the maintenance.
      \item System – The Receptionist goes to the maintenance details, to the section of the maintenance changes and refuses the changes for that maintenance.
    \end{itemize}
      \item Use Case 1.6 – Vehicle Delivery 
    \begin{itemize}
      \item Scenario – The Receptionist delivered the vehicle to the Client.
      \item Objective – Complete the vehicle maintenance process.
      \item System – Sends a report in a PDF format to the client with the information about the maintenance and alters the maintenance request status in the system to conclude. 
    \end{itemize}
          \item Use Case 1.7 – Cancel Mainteannce 
    \begin{itemize}
      \item Scenario – The client does not like the maintenance agreement and wants to retreive the vehicle.
      \item Objective – Cancel the maintenance.
      \item System – The rececionist goes to the maintenance details and cancels the maintenance. 
    \end{itemize}
  \end{itemize}  
  \hfill \break


 The mechanic will be responsible for doing the maintenance in the vehicle, like oil change, tire change, trade vehicle parts, etc. 
 The mechanic use cases are:

  \begin{itemize}
    \item Use Case 2.1 – View to-do list
    \begin{itemize}
      \item Scenario – The mechanic inicialize its shift.
      \item Objective – See tasks to be completed.
      \item System – The mechanic enters the system and encounters a list of tasks assigned to him and to be assign. Each task is accompanied by a description, a priority, a vehicle identification, a set of actions to be performed, and comments from other users. 
    \end{itemize}
    \item Use Case 2.2 – Carry out a vehicle analysis 
    \begin{itemize}
      \item Scenario – A new vehicle needs to be analyzed.
      \item Objective – Confirm the initial analysis of the receptionist and search for additional problems.
      \item System – The mechanic enters a list of tasks that need to be performed in the vehicle. 
    \end{itemize}
    \item Use Case 2.3 – Register tasks completed
    \begin{itemize}
      \item Scenario – A new vehicle needs maintenance.
      \item Objective – Complete maintenance
      \item System – The mechanic selects a list of tasks that he done on the vehicle.
    \end{itemize}
  \item Use Case 2.4 – Do a Vehicle maintenance task
  \begin{itemize}
    \item Scenario – A new vehicle is ready for maintenance.
    \item Objective – The mechanic will do a vehicle maintenance task (oil change, tire change, vehicle wash…).
    \item System – The mechanic starts a task and see a sequence of steps to complete the task. In the final step the mechanic can leave a note.
  \end{itemize}
  \item Use Case 2.5 – Change Task
  \begin{itemize}
    \item Scenario – A created task has the wrong part associated.
    \item Objective – Change the task to the correct part.
    \item System – It is registered a new task change that need to be approved by the workshop manager. In case the inicial budget or the maintenance schedule is altered, the client also needs to be notified and give authorization to change the task.
  \end{itemize}
    \item Use Case 2.6 – Continue Task
  \begin{itemize}
    \item Scenario – The mechanic leave the aplication before finishing the task
    \item Objective – Continue a task previously started.
    \item System – The mechanic can see a tasked paused in the list of the tasks he has to do. By clicking the continue button, the mechanic can continue the task.
  \end{itemize}
\end{itemize}
\hfill \break

The warehouse operator is responsible for managing the dealer's stock and asking for supplies. 
In this case, the use cases are as follows:

\begin{itemize}
  \item Use Case 3.1 – View the different parts that the warehouse possess
  \begin{itemize}
    \item Scenario – The warehouse worker wants to view the quantity of certain parts that the warehouse possesses.
    \item Objective – Show quantitative warehouse information.
    \item System – List of all parts and their quantities that the warehouse possesses. 
  \end{itemize}
  \item Use Case 3.2 – Requesting purchasing service 
  \begin{itemize}
    \item Scenario – The warehouse worker discovers that he has an insufficient number of parts for maintenance or anticipates that this part will be missing soon.
    \item Objective – Request permission to purchase parts from the supplier.
    \item System – The Warehouse Worker will place a purchase order for parts. The system notifies the administrator via the platform and by email requesting authorization to make the purchase. 
  \end{itemize}
    \item Use Case 3.3 – Buy new parts
  \begin{itemize}
    \item Scenario – The warehouse worker contacts the supplier to request new parts.
    \item Objective – Register the information of the new purchase.
    \item System – The warehouse operator introduce to the system the date that the parts will arrive.
  \end{itemize}
  \item Use Case 3.4 – Registration of new parts in the System
  \begin{itemize}
    \item Scenario – The warehouse worker purchased several parts from a supplier.
    \item Objective – Register new parts in the system.
    \item System – The warehouse operator adds to the system the parts that arrived to the warehouse and the date that the purchase is completed.
  \end{itemize}
    \item Use Case 3.5 – Edit Inventory 
  \begin{itemize}
    \item Scenario – The warehouse worker wants to change the part type information.
    \item Objective – Change the part type information.
    \item System – The warehouse operator can change the part type description, location, name or code.
  \end{itemize}
  \item Use Case 3.6 – Create Purchase Delay
  \begin{itemize}
    \item Scenario – A purchase has been delayed.
    \item Objective – Create a purchase delay.
    \item System – The warehouse operator goes to the purchase delay and inserts the new expected arrival date.
  \end{itemize}
\end{itemize}
\hfill \break

The last user of the application is the Workshop Manager. This user is in charge of managing the platform and the dealership. So the main use cases encountered are:

\begin{itemize}
    \item Use Case 4.1 – Assign tasks to the employees
  \begin{itemize}
    \item Scenario – A new maintenance request has been requested.
    \item Objective – Assign and organize tasks to different employees.
    \item System – The Workshop Manager assigns the various tasks of vehicle maintenance to the workshop employees.
  \end{itemize}
  \item Use Case 4.2 – Authorize purchase
  \begin{itemize}
    \item Scenario –  The Workshop Manager received a purchase request.
    \item Objective – Authorize or reject a purchase authorization request.
    \item System – The Workshop Manager can reject or authorize the maintenance request. 
  \end{itemize}
  \item Use Case 4.3 – View history of maintenance performed
  \begin{itemize}
    \item Scenario – The Workshop Manager wants to gather information from recently performed maintenance.
    \item Objective – View information about a specific maintenance that occurred.
    \item System – The Workshop Manager views a list of all maintenance that occurred as well as its details (who carried it out, which parts were removed, the name of the customer, tests carried out, and their results…). 
  \end{itemize}
  \item Use Case 4.4 – Develop statistics
  \begin{itemize}
    \item Scenario – The Workshop Manager wants to gather statistics on the maintenance that was carried out in the last month.
    \item Objective – View information about maintenance over a given period of time.
    \item System – Presentation graphs of the number of parts replaced, number of purchases, total price spent on new parts, remuneration for maintenance, average customer rating, etc.
  \end{itemize}
  \item Use Case 4.5 – Assign roles to employees
  \begin{itemize}
    \item Scenario – A new employee has been hired.
    \item Objective –  Assign roles to new employees.
    \item System – The Workshop Manager assigns to the new employee a certain role and set of permissions.
  \end{itemize}
  %   \item Use Case 4.6 – Authorize task change
  % \begin{itemize}
  %   \item Scenario – The Workshop Manager is notified with a new task change.
  %   \item Objective – Reject or authroize the new task change.
  %   \item System – The Workshop Manager assigns to the new employee a certain role and set of permissions.
  % \end{itemize}
  \item Use Case 4.6 – Add new Task
  \begin{itemize}
    \item Scenario – A task is missing in an on going maintenance.
    \item Objective – Add a new task to the on going maintenance.
    \item System – In the maintenance details the Workshop Manager sees a list of task that he can add to the maintenance. By adding the task, it needs to be validated by the client first.
  \end{itemize}
  \item Use Case 4.7 – Create Employee
  \begin{itemize}
    \item Scenario – A new employee has been hired for the workshop.
    \item Objective – Adding a new employee to the system.
    \item System – The workshop manager fill a form with the name of the new employee, the email, its role, phone number and date of birth .
  \end{itemize}
    \item Use Case 4.8 – Accept/Reject partnership
  \begin{itemize}
    \item Scenario – A new entity wants to work with the dealership to do the maintenance of there's vehicles .
    \item Objective – Associate the entity to the dealerships.
    \item System – Change the status of the request of the partnership request to "accept" or "rejected" depending of the warehouse manager choice.
  \end{itemize}

\end{itemize}
\hfill \break

The client application will only be interacted with by a role of users, the client.
The use cases are listed below:

\begin{itemize}
  \item Use Case 5.1 – View current maintenance status
  \begin{itemize}
    \item Scenario – The Customer wants to find information regarding the vehicle maintenance procedure.
    \item Objective – Display current maintenance status.
    \item System – The system will illustrate all the maintenance steps that the vehicle has already undergone, as well as those that remain to be completed. 
  \end{itemize}
  \item Use Case 5.2 – Notify the customer of the end of maintenance 
  \begin{itemize}
    \item Scenario – Vehicle maintenance has been completed and the customer can now collect the vehicle.
    \item Objective – Notify the user of the end of maintenance.
    \item System – The system will show an SMS/native notification on the customer's cell phone informing that the vehicle is ready to be picked up. 
  \end{itemize}
  \item Use Case 5.3 – Rating of the service provided
  \begin{itemize}
    \item Scenario – The client receives the vehicle and the receptionist completes the maintenance process.
    \item Objective – Get feedback from the client.
    \item System – The system will show a form to the client asking about the service provided. 
  \end{itemize}
    \item Use Case 5.4 – Rating of the service expected
  \begin{itemize}
    \item Scenario – After defining the agreement to the maintenance of the vehicle.
    \item Objective – The Client rating the quality of the service he is expected to receive.
    \item System – The system will show a form to the client asking about the service he is expected to receive. 
  \end{itemize}
\end{itemize}
\hfill \break




\section{Applications Workflow}

After the development of the use cases, a flow chart was designed to understand the users' interaction with the system and each other. The chart is visible in figure \ref{fig:figure2}.

\begin{figure}[h]
  \caption{Use Case Flow Chart of the Client, Receptionist, Mechanic, Warehouse Operator, and Administrator.}
  \centering
  \includegraphics[width=\textwidth]{figs/UseCaseDiagram}
  \label{fig:figure2}
\end{figure}

The system's main flow starts when a Client arrives at the dealership for vehicle maintenance. 
The receptionist gets some input from the client and decides on an initial budget and check-out date agreed with him and inserts the information in the system (Use Case 1.1).
After that, a worker will be responsible for doing the analysis of the vehicle. 
He will add to the system all the problems he finds and the necessary parts that need to be replaced (Use Case 2.2 and 2.3).


If the budget to complete the work or the expected time changes, an alert is sent to the receptionist to inform the Client (Use Case 1.4). 
The result of this interaction must be "authorize the changes and continue the work", "not authorize the changes but continue as initially agreed" or "not authorizing the changes and wanting to check out the vehicle".
In the last case, the vehicle is delivered and the app will request the client to rate the service (Use Case 1.2 and 5.3). 
In any other case, the maintenance process continues. The only difference is the maintenance process information is alterated in the not authorized case.

After this step, the admin will receive a notification of the new tasks and will assign them to each worker (Use Case 4.5).
From there on, the maintenance process can or can not require the need for a vehicle part to be replaced. 
In the negative case the vehicle goes to the responsible mechanic to do the oil change, tire change, wash, etc (Use Case 2.6). 
In the affirmative case, the Warehouse Operator will check if the parts the mechanic requested are available in stock. 
If it does, the operator accepts the request and delivers the parts to the mechanic, who will replace the parts of the vehicle, deliver the damaged parts to the warehouse, and do the maintenance (Use Case 2.4, 2.5 and 2.6). 
If it does not, the operator needs to buy new parts from the supplier. 
In this case, he initiates a new requesting purchase that can be authorized by the admin. 
In the optimal case, the admin accepts the purchase, the operator buys the new parts, registers them in the system, and delivers them to the mechanic. 
In the worst scenario, the admin rejects the purchase and the mechanic needs to request new parts and restart the process (Use Case 2.3).    

Finally, when the vehicle maintenance is finished, the mechanic inserts into the system all operations and tests carried out as well as their results (Use Case 2.7). 
At once, the system notifies the customer that the vehicle is ready for the check out and, when the receptionist delivers the vehicle to the Customer, the client application asks to rate the system (Use Case 5.2, 1.2 and 5.3).

There are also a few secondary flows visible in the figure \ref{fig:figure2}. 
These flows are listed below:
\begin{itemize}
  \item The client enters the application to check the status of the maintenance (Use case 5.1);
  \item The mechanic enters the application to view the task he has to do (Use Case 2.1); 
  \item The Warehouse Operator enters the application to visualize the diverse parts and components in the warehouse (Use case 3.1); 
  \item The Admin enters the application to assign roles and/or permissions to the employees (Use Case 4.4); 
  \item The Admin enters the application to gather information about a maintenance performed and see statistics about that information (Use Case 4.2 and 4.3); 
\end{itemize}
 



\section{Database} 


\begin{figure}[h]
  \caption{Database general diagram.}
  \centering
  \includegraphics[width=\textwidth]{figs/dbDiagrams/DbDiagram_Full}
  \label{fig:figure2}
\end{figure}

In this section i will describe the dabase structure i create to solve this problem.
To solve this problem i create a structure that devides in 5 sections:
- Maintenance & Tasks
- Parts & Inventory
- Purchasing
- Services
- Vehicles & Owners
- Users

The section users was provided by the ligthmobie's dabase structure and i used it to create users, create the roles of the users and create the group of users (client, rececionist, warehouse manager,...).



\subsection{Maintenance & Tasks} 





