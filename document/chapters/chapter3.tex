\chapter{Methodology}%
\label{chapter:methodology}

\begin{introduction}
A short description of the chapter.

A memorable quote can also be used. asas
\end{introduction} 


\section{Arquitecture}

The architecture of the system will have a MVC pattern. 
The choice of this patter is substantiated by it's simple modificability [25], which is advantages in this situation since project is still not fully stablished and may receive some suggestion from the company along the way;
it's easier way to debug and test the aplication; it's efficient development process; and my own familiaryty with the pattern. [25]

The choice of the tecnology to build the application will be the asp net core mvc. 
I chose this tecnology because it's facility to integrate with the mysql database with the package Entity Framework Core. 
This package provides a \ac{ORM} system that allows the developer to work with the database without writing SQL queries. 

\section{Service Quality}
The concept of service quality is well-documented, with the GAP model and SERVQUAL model being among the most recognized frameworks. SERVQUAL, in particular, provides a multidimensional approach for comparing consumers’ perceptions of service quality against their expectations. It emphasizes five core dimensions [12]:

\begin{itemize}
    \item Reliability – Consistently delivering services as promised.
    \item Responsibility – Ensuring accuracy, a willingness to assist, and timely service.
    \item Assurance – Demonstrating employee knowledge, politeness, and trustworthiness.
    \item Empathy – Providing personalized service and treating customers as individuals.
    \item Tangible Elements – Highlighting the physical aspects and material representation of the service.
  \end{itemize}

The GAP model, precedent from the SERVQUAL, also measures service quality by identifying the difference between customer expectations and actual perceptions. These gaps include discrepancies between customer expectations and management's understanding, management’s perceptions and service specifications, service specifications and actual delivery, actual delivery and communicated services, and expected and delivered service. [13]

\section{Service Management for MAS Motors LLC}