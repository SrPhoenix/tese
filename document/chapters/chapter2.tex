\chapter{Literature Review}%
\label{chapter:literatureReview}

\begin{introduction}
A short description of the chapter.

A memorable quote can also be used. asas
\end{introduction} 


To acquire information about this topic I search information on the platform scopus and google academics. 
In Scopus, I used multiple queries, the first one was Vehicle AND Management AND dealerships and found 24 papers that were related to this topic, but on further examination, those papers were reduced to Y because… 
In google scholar i found 3 papers about the electric vehicle, which is the main vehicle focus of this dissertation. 
This paper talks about the impact of the electric vehicle in china. 
They are important because if we can understand this market in another country has a turning point of interest, the LightMobie can expand its business (the bike sharing) to that country, which in turn will surge the need for dealerships to arise.  


\section{Background}

The LightMobie is a company stablished in Águeda that provides a diverse set of products in the shared mobility sector. [14]
The company also provides a software system to integrate with their bicycles and stations and vehicle maintenance. [14]
Currently, the company is in step of renewal of the system to a new solution of the bicycles and stations and is building a platform to interact and manage that system.
This platform is able to visualize current and past information of the vehicle, stations and users and there interaction with each other.
Addicionally, alerts triggered by the equipment are also visible in the dashboard and the user can elaborate statistics on this data.

\section{Vehicle Maintenance}

Delivering vehicle maintenance services involves numerous activities, each crucial for maintaining service quality and ensuring customer satisfaction. 
Any deviation from these activities can negatively impact quality, leading to dissatisfaction. 
The dissatisfaction of a client reduces their loyalty to company, wich can lead to a decrease in the number of clients and a decrease in revenue. [11]
To mitigate this Quality Control must supervise every stage of the process—from scheduling appointments to repairs and final vehicle delivery. [11]

The macro-level flow of a vehicle maintenance or repair service is illustrated in the figure BELOW. 

The first step of the process starts with a client interacting with the garage to schedule a maintenance or repair. 
After that, the client goes to the garage and the receptionist receives the client and the vehicle.
Then the mechanic will perform the maintenance or repair on the vehicle.
And, Finally, the client will receive the vehicle back from the garage.

To ensure quality in the first step the receptionist must understand the fill capacity of the garage and the time to complete the job. 
This step is very important because a mistake can lead to a break of promisses for the organization. [11]
The second step the recepcionist and service advisor, when receiving the vehicle, must do a visual confirmation of the vehicle condition. [11]
Explain to the client the services of the garage and their prices. Agree with the client the services to be performed and the time to complete the job. 
Then recepcionist insert into the system this information. [11]

Following beguins the third step and most important fase, the mechanic will perform the maintenance or repair on the vehicle. 
In this process, the quality control must supervise all the proccess to ensure that the job is done correctly. [11]
This includes the repair process, extra work, final tests, vehicle wash, service report and promissed date and delivery. [11]
To accomplish that the use of a checklist is recommended to ensure precision and accuracy at each step. [11]

Finally, the last step is the delivery of the vehicle to the client. 
In here  the admin must review the work done and the final price of the service, to avoid unnecessary payments from the client and not paying the workers for the done work. [11]

with this the a vehicle maintenance is completed with quality. 
But how do you measure the quality of the service?
I'll address that in the next section.

\section{Service Quality}
The concept of service quality is well-documented, with the GAP model and SERVQUAL model being among the most recognized modules. 
SERVQUAL, in particular, provides a multidimensional approach for comparing consumers’ perceptions of service quality against their expectations. 
It emphasizes five core dimensions [12]:

\begin{itemize}
  \item Tangibles – Physical facilities, equipment, and appearance of personnel.
  \item Reliability – Ability to consistently deliver services as promised.
  \item Responsiveness – Willingness to assist the customer and proactivity.
  \item Assurance – Demonstrate courtesy and knowledge and inspire trust and confidence.
  \item Empathy – Caring and treat customers as individuals.
\end{itemize}

To measure the quality of the service one needs to do two intervies. One before the service is done and another after the service is done. [27]
Each interview is composed of 22 questions that are divided into the five dimensions of the SERVQUAL model and with the answers interval number of reference, like between 0 and 5. [27] [13]
The difference between the two interviews is the quality of the service. [27] [13] [12]

In the paper of [13] the authors used the SERVQUAL model to measure the quality of the service of a car dealership.
...

FALAR DO Measuring after-sales service quality in automobile retails: An application of the SERVQUAL instrument


The GAP model, precedent from the SERVQUAL, also measures service quality by identifying the difference between customer expectations and actual perceptions. 
These gaps include discrepancies between customer expectations and management's understanding, management’s perceptions and service specifications, service specifications and actual delivery, actual delivery and communicated services, and expected and delivered service. [13]

https://forms.app/en/blog/gap-model-of-service

\section{Service Management for MAS Motors LLC}

\section{Increase Clients Satisfaction in Vehicle Maintenance}