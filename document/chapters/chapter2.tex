\chapter{Literature Review}%
\label{chapter:literatureReview}

\begin{introduction}
A short description of the chapter.

A memorable quote can also be used. asas
\end{introduction} 


\section{Background}
The LightMobie is a company stablished in Águeda that provides a diverse set of products in the shared mobility sector. [14]
The company also provides a software system to integrate with their bicycles and stations and vehicle maintenance. [14]
Currently, the company is in step of renewal of the system to a new solution of the bicycles and stations and is building a platform to interact and manage that system.
This platform ....

\section{Vehicle Maintenance}

Delivering vehicle maintenance services involves numerous activities, each crucial for maintaining service quality and ensuring customer satisfaction. Any deviation from these activities can negatively impact quality, leading to dissatisfaction. To mitigate this, robust quality and inspection systems are indispensable, with Quality Control playing a pivotal role in overseeing the repair process and driving benefits for both clients and the company. [11]
The macro-level flow of a vehicle maintenance or repair service is illustrated in the figure BELOW. An effective organizational structure is fundamental to achieving quality outcomes and customer satisfaction. Key roles include receptionists, service advisors, workshop managers, mechanics, technical support staff, and parts department personnel. At the strategic level, top management is responsible for strategic planning, aligning organizational goals, and establishing quality policies to achieve defined quality objectives. [11]
To meet customer expectations for high-quality repairs, Quality Control must supervise every stage of the process—from scheduling appointments to repairs and final vehicle delivery. In addition to active oversight, the use of a comprehensive “checklist” ensures precision and accuracy at each step. [11]

\section{Service Quality}
The concept of service quality is well-documented, with the GAP model and SERVQUAL model being among the most recognized frameworks. SERVQUAL, in particular, provides a multidimensional approach for comparing consumers’ perceptions of service quality against their expectations. It emphasizes five core dimensions [12]:

\begin{itemize}
    \item Reliability – Consistently delivering services as promised.
    \item Responsibility – Ensuring accuracy, a willingness to assist, and timely service.
    \item Assurance – Demonstrating employee knowledge, politeness, and trustworthiness.
    \item Empathy – Providing personalized service and treating customers as individuals.
    \item Tangible Elements – Highlighting the physical aspects and material representation of the service.
  \end{itemize}

The GAP model, precedent from the SERVQUAL, also measures service quality by identifying the difference between customer expectations and actual perceptions. These gaps include discrepancies between customer expectations and management's understanding, management’s perceptions and service specifications, service specifications and actual delivery, actual delivery and communicated services, and expected and delivered service. [13]

\section{Service Management for MAS Motors LLC}