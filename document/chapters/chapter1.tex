\chapter{Introduction}%
\label{chapter:introduction}

\begin{introduction}
In this chapter I will present the relevance of vehicle maintenance in the current time. Addicionally, I will address the objectives of this dissertation and the strucuture of this document.
\end{introduction} 


\section{Motivation}

In the past few years, the vehicle sharing market has seen a significant growth, due to the global interest in sustainability and environmental issues. \cite{bike_data_businessresearch}
The bike-sharing market, in particular, has gained popularity from the governments worldwide \cite{Clercq2023} \cite{Cerro2024} \cite{European_declararion_on_cycling} \cite{cohesionOpenData}, which are investing in cycling infrastructure such as cycling lanes, secure parking facilities, bicycle production and repair industries \cite{Cerro2024} \cite{European_declararion_on_cycling} \cite{cohesionOpenData}.
This attention to this sector raise the need for efficient vehicle maintenance solutions.

The fast progress of tecnology can prove to be a valuable ally in this matter. 
From the use of IoT devices with machine learning algorithms to monitor the vehicles health \cite{Vasavi2021}, 
to the use of task management software to manage the maintenance tasks \cite{MARS_MOTORS}, 
there are a large range of solutions that can be implemented to improve the efficiency of the maintenance process.

This dissertation presents a software solution that facilitates and organizes the work at the dealerships of the company LightMobie. 
It will integrate with the fleet management platform, a project that is being in development since a few years. 
This project is being developed today by a research team from  \ac{IEETA} at the University of Aveiro with the presence of the company LightMobie.
It is a platform to manage the vehicle sharing system of an entity like hotels, city hall, etc.  

\section{Objectives and problems}

The work in a vehicle maintenance service provider may be organize by manual input wtih applications like Visual Basic, Escel, Word and paper documents. \cite{MARS_MOTORS} 
This may introduce accidental human erros, which in turn lead to the managers' intuintion to make a decision based on the data. \cite{MARS_MOTORS}
This type of apporach may not be very prejudice in a small business, but the continuous growth of the company requires a more modern method to handle the huge amount of work and achieve continuous success. \cite{MARS_MOTORS}

This dissertation will focus on the development of a web application that facilitates and increase the performence of the work at the dealership.
To accomplish this, the application will allow, simultaneously, to share information, manage and control a shared vehicle dealership network.
It will also interact with a factory data warehouse to gather information and to store data generated from the maintenance process. 

This presents a significant challenge related to the users that work at the dealerships and garages. 
This users may be already acustumated to their own systems or there manual labor of work.  
The introduction of a new system may be a challenge to the users, so the system must be user-friendly, easy to use and provide a significant value in their work.

\section{Structure of the Thesis}

This dissertation is organized in five chapters. The chapter one explain the relevance of the topic and the objectives of this dissertation.
The chapter two presents the literature review about vehicle maintenance, service quality and a software solution implementation. 
Also in this chapter, I present a small background of the LightMobie Bike Sharing System and a analysis of the available arquitecture.
The chapter three presents the requirements of the system and the use cases that will be implemented.
The chapter four presents the work plan.
The chapter five presents the conclusion of the dissertation. 

