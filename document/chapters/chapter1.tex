\chapter{Introduction}%
\label{chapter:introduction}

\begin{introduction}
This chapter presents the global relevance of vehicle maintenance. Additionally, it addresses the objectives of this dissertation and the strucuture of this document.
\end{introduction} 


\section{Motivation}

The vehicle sharing market has seen a significant growth since 2021 due to the global interest in sustainability and environmental issues. ~\cite{cohesionOpenData} ~\cite{bike_data_businessresearch}
The bike-sharing market, in particular, has gained popularity from the governments worldwide, which are investing in cycling infrastructure such as cycling lanes, secure parking facilities, bicycle production and repair industries ~\cite{Clercq2023} ~\cite{Cerro2024} ~\cite{European_declararion_on_cycling} ~\cite{bike_data_businessresearch} ~\cite{cohesionOpenData}.
This attention to this sector raise the need for vehicle maintenance infrastructure that may be outdated and inneficient. ~\cite{MAS_MOTORS}
This low quality of service can be to a significant decrease on the layalty of the customers, which is the main source of income. ~\cite{Setting_the_after_sale_process}
This dissertation proposes a solution to solve this problem.

The fast progress of technology can prove to be a valuable ally in this matter. 
From the use of IoT devices with machine learning algorithms to monitor the vehicles health ~\cite{Vasavi2021}, 
to the use of task management software to manage the maintenance tasks ~\cite{MAS_MOTORS}, 
there are a large range of solutions that can be implemented to improve the efficiency of the maintenance process.

This dissertation presents a software solution that facilitates and organizes the work at the dealerships of the company LightMobie. 
It will integrate with the fleet management platform, a project that is in development since a few years. 
This project is being developed today by a research team from  \ac{IEETA} at the University of Aveiro with the presence of the company LightMobie.
It is a platform to manage the vehicle sharing system of an entity like hotels, city hall, etc.  

\section{Objectives and problems}

The work in a vehicle maintenance service provider may be organize by manual input wtih basic applications. ~\cite{MAS_MOTORS} 
This may introduce accidental human errors, which in turn lead break of promisses or unsatisfaction from the customer. ~\cite{MAS_MOTORS} ~\cite{Setting_the_after_sale_process}
This type of apporach may not be very prejudice in a small business, but the continuous growth of the company requires a more modern method to handle the huge amount of work and achieve continuous success. ~\cite{MAS_MOTORS}

This dissertation will focus on the development of a web application that facilitates and increases the performance of the work at the dealership.
To accomplish this, the application will allow, simultaneously, to share information, manage and control a shared vehicle dealership network.
It will also interact with a factory data warehouse to gather information and to store data generated from the maintenance process. 

This dissertation presents a significant challenge related to the users that work at the dealerships and garages. 
This users may be already accustomed to their own systems or there manual labor of work.  
The introduction of a new system may be a challenge to the users, so the system must be user-friendly, easy to use and provide a significant value in their work.

\section{Structure of the Thesis}

This dissertation is organized in five chapters. The chapter one explain the relevance of the topic and the objectives of this dissertation.
The chapter two presents the literature review about vehicle maintenance, service quality and software solution implementation. 
% This chapter also presents a small background of the LightMobie Bike Sharing System and a analysis of the available architecture.
The chapter three describes the requirements of the system and the use cases that will be implemented.
The chapter four illustrates the work plan.
The chapter five shows the conclusion of the dissertation. 

